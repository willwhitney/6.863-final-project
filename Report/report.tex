\documentclass[11pt]{article}
\usepackage[margin=3cm]{geometry}
\renewcommand*\abstractname{Summary}

\begin{document}

\title{Solving Semantic Analogy Problems Using ConceptNet}
\author{Mika Braginsky and Will Whitney}
\maketitle

\begin{abstract}
We implemented a system for solving SAT analogy problems, using ConceptNet. The system attempts to find the relationship between the target pair of words (such as \emph{mason}:\emph{stone}) and each option pair (such as \emph{teacher}:\emph{chalk}, \emph{carpenter}:\emph{wood}, \emph{soldier}:\emph{gun}, \emph{photograph}:\emph{camera}, \emph{book}:\emph{word}), scores the similarity of each option's relationship to the target's relationship, and selects the option with the highest score. On a dataset of 374 questions, it achieves an accuracy rate of 28.1\%.
\end{abstract}

\section{Problem Overview}
\textit{TODO: define the problem, give examples, explain relevance to intelligence}

\section{Previous Work}
\textit{TODO: summarize previous work, show results table}

\section{Approach}
\textit{TODO: explain our approach, connect to human strategies, describe ConceptNet}

\section{Implementation}
\textit{TODO: explain our imeplementation: queries to ConceptNet, parallelization, finding paths, similarity metric}

\section{Results}
\textit{TODO: show our results, discuss error types}

\section{Further Work}
\textit{TODO: give options of ways this could be improved/extended}

\end{document}
